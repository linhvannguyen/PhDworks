% Auteur de la thèse : prénom (1er argument obligatoire), nom (2e argument
% obligatoire) et éventuel courriel (argument optionnel). Les éventuels accents
% devront figurer et le nom /ne/ doit /pas/ être saisi en capitales.
\author[linh.van.nguyen@hotmail.com]{Linh Van}{Nguyen}
%
% Titre de la thèse dans la langue principale (argument obligatoire) et dans la
% langue secondaire (argument optionnel).
\title[Reconstruction fine de champ de vitesses d'un écoulement turbulent à partir de mesures faiblements résolues]{Reconstruction of finely resolved velocity fields in turbulent flows from low resolution measurements}
%
% Sous-titre de la thèse dans la langue principale (argument obligatoire) et
% dans la langue secondaire (argument optionnel). Facultatif.
%\subtitle[...]{...}
%
% Champ disciplinaire dans la langue principale (argument obligatoire) et dans
% la langue secondaire (argument optionnel).
\academicfield[Mécanique]{Mechanics}
%
% Spécialité dans la langue principale (argument obligatoire) et dans la langue
% secondaire (argument optionnel). Facultatif.
\speciality[Turbulence]{Turbulence}
%
% Date de la soutenance, au format {jour}{mois}{année} donnés sous forme de
% nombres.
\date{28}{9}{2016}
%
% Sujet pour les méta-données du PDF. Facultatif.
\subject[Reconstruction fine de champ de vitesses d'un écoulement turbulent à l'aide de données expérimentales faiblements résolues]{Reconstruction of finely resolved velocity vector fields in turbulent flows from low resolution measurements}
%
% Nom (argument obligatoire) du PRES. Facultatif.
%\pres[logo=./images/logos/pres]{Université Lille Nord de France}
\pres[logo=./images/logos/labs]{LML, CRIStAL}
%
% Nom (argument obligatoire) de l'institut (principal en cas de cotutelle).
\institute[logo=./images/logos/lille1.png,url=http://www.univ-lille1.fr/]{Université Lille 1}

%
% En cas de cotutelle (normalement, seulement dans le cas de cotutelle
% internationale), nom (argument obligatoire) du second institut. Facultatif.
%\coinstitute[logo=./images/logos/ECLILLE,url=http://www.ec-lille.fr/en/index.html/]{École Centrale de Lille}

%\coinstitute[logo=./images/logos/CNRS,url=http://www.cnrs.fr/]{CNRS}

%
% Nom (argument obligatoire) de l'école doctorale. Facultatif.
\doctoralschool[url=http://edspi.univ-lille1.fr/]{SPI $072$}
%
% Nom (1er argument obligatoire) et adresse (2e argument obligatoire) du
% laboratoire (ou de l'unité) où la thèse a été préparée, à utiliser /autant de
% fois que nécessaire/.

\laboratory[
logo=./images/logos/LML.jpg,
logoheight=1.25cm,
telephone=(33)(0) 3 20 33 71 52,
fax=(33)(0) 3 20 33 71 53,
email=lml@univ-lille1.fr,
url=http://lml.univ-lille1.fr/lml/
]{LML}{%
  Laboratoire de Mécanique de Lille \\
  FRE 3723            \\
  Bvd Paul Langevin         \\
  59655 Villeneuve d'Ascq,CEDEX   \\
  France}

\laboratory[
logo=./images/logos/Cristal.png,
logoheight=1.25cm,
%telephone=33 (0)3 20 67 60 70,
%fax=33 (0)3 20 33 54 18,
%email=lagis.direction@cnrs.fr,
url=http://cristal.univ-lille.fr/
]{CRIStAL}{%
  Centre de Recherche en Informatique, \\
  Signal et Automatique de Lille  \\
  UMR 9189 CNRS            \\
  CS 20048 					\\
  59651 Villeneuve d'Ascq CEDEX \\ 
  France}
  
%
% Membres du jury, saisis au moyen des commandes \supervisor, \cosupervisor,
% \comonitor, \referee, \committeepresident, \examiner, \guest, à utiliser
% /autant de fois que nécessaire/ et /seulement si nécessaire/. Toutes basées sur
% le même modèle, ces commandes ont 2 arguments obligatoires, successivement les
% prénom et nom de chaque personne. Si besoin est, on peut apporter certaines
% précisions en argument optionnel, au moyen des clés suivantes :
% - « professor », « seniorresearcher », « mcf », « mcf* »,
%   « juniorresearcher », « juniorresearcher* » (qui peuvent ne pas prendre de
%   valeur) pour stipuler le corps auquel appartient la personne ;
% - « affiliation » pour stipuler l'institut auquel est affiliée la personne.
%
\supervisor[juniorresearcher*,affiliation=CNRS]{Jean-Philippe}{Laval}
\supervisor[mcf*,affiliation=École Centrale de Lille]{Pierre}{Chainais}
\referee[professor,affiliation=Grenoble-INP]{Olivier}{Michel}
\referee[seniorresearcher,affiliation=INRIA]{Etienne}{Mémin}
%\committeepresident[professor,affiliation=Université de Rouen]{Danaila}{Luminita}
\examiner[professor,affiliation=Université de Rouen]{Danaila}{Luminita}
\examiner[professor,affiliation=INP-ENSEEIHT Toulouse]{Nicolas}{Dobigeon}
%
% Mention du numéro d'ordre de la thèse (s'il est connu, ce numéro est
% à spécifier en argument optionnel). Facultatif.
\ordernumber[to be defined]
%
% Préparation des mots clés dans la langue principale (1er argument) et dans la
% langue secondaire (2e argument)
%%%%%%%%%%%%%%%%%%%%%%%%%%%%%%%%%%%%%%%%%%%%%%%%%%%%%%%%%%%%%%%%%%%%%%%%%%%%%%%
\keywords{reconstruction, turbulence, dictionary learning, non-local means, Bayesian fusion, regression, small scales}{reconstruction, turbulence, apprentissage de dictionnaire, moyenne non-locale, fusion bayésienne, régression, petites échelles}
